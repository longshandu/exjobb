\theabstract{Video content is the dominant traffic type on mobile networks today and this portion is only expected to increase in the future. In this thesis we investigate ways of reducing bit rates for adaptive streaming applications in the latest video coding standard, H.265 / High Efficiency Video Coding (HEVC).

The current models for offering different-resolution versions of video content in a dynamic way, so called \textit{adaptive streaming}, require either large amounts of storage capacity where full encodings of the material is kept at all times, or extremely high computational power in order to regenerate content on-demand.

Guided transcoding aims at finding a middle-ground were we can store and transmit less data, at full or near-full quality, while still keeping computational complexity low. This is achieved by shifting the computationally heavy operations to a preprocessing step where so called \textit{side-information} is generated. The side-information can then be used to quickly reconstruct sequences on-demand -- even when running on generic, non-specialized, hardware.

Two method for generating side-information, pruning and deflation, are compared on a varying set of standardized HEVC test sequences and the respective upsides and downsides of each method are discussed.}
